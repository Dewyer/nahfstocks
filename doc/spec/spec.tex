%!suppress=Unicode
\documentclass{article}[12pt,a4paper]
\usepackage[utf8]{inputenc}
\usepackage{caption}
\usepackage{amsmath}
\usepackage{hyperref}
\usepackage{csquotes}
\usepackage[T1]{fontenc}
\usepackage{booktabs}
\usepackage{subfiles}

\hypersetup{
    colorlinks=true,
    linkcolor=blue,
    filecolor=magenta,
    urlcolor=cyan,
}

\title{Részvénypiac Szimuláció}
\author{Rátki Barnabás}
\date{2021.03.29}

\newcommand{\tbs}{\textbackslash}
\newcommand{\tc}{\textasciicircum}

\begin{document}
    \maketitle


    \section{Rövid összefoglaló a problémáról}\label{sec:description}
    A program célja egy képzeletbeli részvénypiac szimulációja. \\
    \textit{Megjegyzés: Nem cél a valós rendszer közeli modellezése, mert ez valószínűleg lehetetlen, a feltételezések, amikre a szimuláció épül nem a valóságon alapulnak és lényeges egyszerűsítéseket tartalmaznak.}\\
    Az elképzelt piacunkban két különböző szereplő van, cégek illetve kereskedők.
    A részvények tulajdonos-cseréjét pedig egy központosított piac biztosítja.
    Az összes többi piaci körülményekben változást elérő hatást véletlenszerű események fogják generálni.
    A szimulációban arra vagyunk kiváncsiak, hogy az adott beállításokkal és random \textit{seed}-el milyenek lesznek az árfolyamok grafikonjai a különböző cégeknél és mennyi tőkéjük lesz a kereskedőknek.
    A kereskedők kezdetben random generált beállításokkal működnek, amik egyedivé teszik az akcióikat és viselkedésüket de összeségében az a céljuk, hogy a maximalizálják profitjaikat.
    A cégek hasonlóan random generáltak, és a szimuláció elején fix mennyiségű részvényt bocsájtanak ki és igyekszenek maximalizálni az így beszedett pénz mennyiségét.
    A rendszer ciklusokban működik, minden ciklusban létrejöhetnek események, illetve random aktivált kereskedők kezelhetik befektetéseik.
    A piac egy úgynevezett "Aukciós Piac" elvén működik, ahol a kereskedők ajánlatainak egyezése esetén történnek eladások és vételek.


    \section{Bemenetek}
    A program egy parancsoros interfacen keresztül használható aminek a standard bemeneten lehet megadni különböző beállításokat
    Ezeket egy egyszerű kulcs-érték pár listaként várja a következő formában:
    \textit{KULCS=ÉRTÉK} a párok pedig valamilyen "whitespace" karakterrel kell, hogy elválasztva legyenek.
    A felhasználó dupla "whitespace" karakter bevitelével tudja jelezni, hogy több beállítást nem szeretne megadni.


    \section{Beállítások}
    \begin{center}
        \begin{tabular}{ | p{5.5cm} || p{5.5cm} | }
            \hline
            \textbf{Kulcs / Tipus}               & Leírás                                                                                \\
            \hline
            \textbf{INTERACTIVE\_MODE (Bool)}    & A program interaktív módban vagy limit* módban fusson.                                \\
            \hline
            \textbf{CYCLE\_LIMITS (Int)}         & Maximum mennyi ciklusig tarthat a szimuláció.                                         \\
            \hline
            \textbf{SEED (Int)}                  & Mi legyen a random \textit{seed**}                                                    \\
            \hline
            \textbf{LOG\_LEVEL (Int)}            & Mennyire legyen a standard kimenetre való naplózás részletes.                         \\
            \hline
            \textbf{TRADER\_COUNT (Int)}         & Hány random generált kereskedők legyen.                                               \\
            \hline
            \textbf{COMPANY\_COUNT (Int)}        & Hány random generált cég legyen.                                                      \\
            \hline
            \textbf{EARNINGS\_CYCLES (Int)}      & Mennyi ciklusonként legyen "earnings" jelentése a cégeknek átlagosan.                 \\
            \hline
            \textbf{DIVIDEND\_CYCLES (Int)}      & Mennyi ciklusonként legyen dividáns fizetés átlagosan.                                \\
            \hline
            \textbf{TRADER\_MONEY (Int)}         & Mennyi legyen a kereskedők átlag kezdő vagyona.                                       \\
            \hline
            \textbf{TRADER\_INCOME (Int)}        & Mennyi legyen a kereskedők átlag keresete.                                            \\
            \hline
            \textbf{MEDIAN\_IPO (Int)}           & Mennyi legyen a medián kezdetleges részvénykibocsájtás részvényenkénti ára.           \\
            \hline
            \textbf{PRICE\_SAMPLING\_RATE (Int)} & A szimuláció hűny ciklusonként mintavételzze a cégek árfolyamát. (Kimenet felbontása) \\
            \hline
        \end{tabular}\\
        \textit{* - Limit módban a program, menü megjelenítése nélkül CYCLE\_LIMITS-nyi cikluson keresztül fut aztán kilép.} \\
        \textit{** - A programban ez az egyetlen véletlenszerű forrás, egyébként azonos seedekkel determinisztikusan működik.}
    \end{center}
    Az összes beállítás opcionális és implementáció függő alapértelmezett értékekkel rendelkezik.
    Az implementáció támogathat még ezeken kívül más beállításokat is.
    Illetve a program "--help" parancssori argomentummal történő futtatására kiírja a támogatott beállításait és egy rövid használati útmutatót.


    \section{Interaktív mód}
    Interaktív módban való futtatáskor a szimulácó aktuális állapota egy menü segítségével megfigyelhető.
    A továbbiakban az ebből a menüből elérhető funkciókat ismertetem:

    \subsection{Szimuláció megállítása/elindítása}
    El lehet indítani a szimulációt, a ciklusok addig fognak futni ameddig a felhasználó be nem ír valamit a standard bemenetre, illetve előre is meg lehet adni, hogy hány ciklust menjen mielőtt újra megállna.
    A programot véglegesen is le lehet állítani.

    \subsection{Statisztikák lekérése}
    Meg lehet tekinteni aktuális statisztikákat a szimulációról, mennyi pénz kering a rendszerben, melyik a legnagyobb cék illetve, ki a leggazdagabb kereskedő, hányadik ciklusban van.

    \subsection{Cégek listázása}
    Ki lehet listázni a cégeket, szimbólumaikkal, teljes nevükkel illetve jelenlegi árfolyamaikkal.

    \subsection{Cégek részletes adatai}
    Le lehet kérni egy cég részletes adatait, rejtett tulajdonságaival és visszamenőleges árfolyam adatokkal szimbólum alapján.

    \subsection{Kereskedők listázása}
    Ki lehet listázni a kereskedőket, nevükkel, egyenlegeikkel, portfólióik méretével illetve egyedi azonosítójaikkal.

    \subsection{Kereskedők részletes adatai}
    Le lehet kérni egy kereskedő részletes adatait, beállításait, összes nyitott pozícióját a kereskedő egyedi azonosítója alapján.


    \section{Kimenetek}
    A szimuláció végeztével, a program az elmentett árfolyam, cég és kereskedő információt kiírja JSON formátumban.
    A konkrét formátum implementáció függő de alapvetőn hasonló adatokat tartalmaz, mint ami az interaktív menüben elérhető, csak teljesen részletesen.
\end{document}